% Options for packages loaded elsewhere
\PassOptionsToPackage{unicode}{hyperref}
\PassOptionsToPackage{hyphens}{url}
\documentclass[
]{article}
\usepackage{xcolor}
\usepackage[margin=1in]{geometry}
\usepackage{amsmath,amssymb}
\setcounter{secnumdepth}{-\maxdimen} % remove section numbering
\usepackage{iftex}
\ifPDFTeX
  \usepackage[T1]{fontenc}
  \usepackage[utf8]{inputenc}
  \usepackage{textcomp} % provide euro and other symbols
\else % if luatex or xetex
  \usepackage{unicode-math} % this also loads fontspec
  \defaultfontfeatures{Scale=MatchLowercase}
  \defaultfontfeatures[\rmfamily]{Ligatures=TeX,Scale=1}
\fi
\usepackage{lmodern}
\ifPDFTeX\else
  % xetex/luatex font selection
\fi
% Use upquote if available, for straight quotes in verbatim environments
\IfFileExists{upquote.sty}{\usepackage{upquote}}{}
\IfFileExists{microtype.sty}{% use microtype if available
  \usepackage[]{microtype}
  \UseMicrotypeSet[protrusion]{basicmath} % disable protrusion for tt fonts
}{}
\makeatletter
\@ifundefined{KOMAClassName}{% if non-KOMA class
  \IfFileExists{parskip.sty}{%
    \usepackage{parskip}
  }{% else
    \setlength{\parindent}{0pt}
    \setlength{\parskip}{6pt plus 2pt minus 1pt}}
}{% if KOMA class
  \KOMAoptions{parskip=half}}
\makeatother
\usepackage{color}
\usepackage{fancyvrb}
\newcommand{\VerbBar}{|}
\newcommand{\VERB}{\Verb[commandchars=\\\{\}]}
\DefineVerbatimEnvironment{Highlighting}{Verbatim}{commandchars=\\\{\}}
% Add ',fontsize=\small' for more characters per line
\usepackage{framed}
\definecolor{shadecolor}{RGB}{248,248,248}
\newenvironment{Shaded}{\begin{snugshade}}{\end{snugshade}}
\newcommand{\AlertTok}[1]{\textcolor[rgb]{0.94,0.16,0.16}{#1}}
\newcommand{\AnnotationTok}[1]{\textcolor[rgb]{0.56,0.35,0.01}{\textbf{\textit{#1}}}}
\newcommand{\AttributeTok}[1]{\textcolor[rgb]{0.13,0.29,0.53}{#1}}
\newcommand{\BaseNTok}[1]{\textcolor[rgb]{0.00,0.00,0.81}{#1}}
\newcommand{\BuiltInTok}[1]{#1}
\newcommand{\CharTok}[1]{\textcolor[rgb]{0.31,0.60,0.02}{#1}}
\newcommand{\CommentTok}[1]{\textcolor[rgb]{0.56,0.35,0.01}{\textit{#1}}}
\newcommand{\CommentVarTok}[1]{\textcolor[rgb]{0.56,0.35,0.01}{\textbf{\textit{#1}}}}
\newcommand{\ConstantTok}[1]{\textcolor[rgb]{0.56,0.35,0.01}{#1}}
\newcommand{\ControlFlowTok}[1]{\textcolor[rgb]{0.13,0.29,0.53}{\textbf{#1}}}
\newcommand{\DataTypeTok}[1]{\textcolor[rgb]{0.13,0.29,0.53}{#1}}
\newcommand{\DecValTok}[1]{\textcolor[rgb]{0.00,0.00,0.81}{#1}}
\newcommand{\DocumentationTok}[1]{\textcolor[rgb]{0.56,0.35,0.01}{\textbf{\textit{#1}}}}
\newcommand{\ErrorTok}[1]{\textcolor[rgb]{0.64,0.00,0.00}{\textbf{#1}}}
\newcommand{\ExtensionTok}[1]{#1}
\newcommand{\FloatTok}[1]{\textcolor[rgb]{0.00,0.00,0.81}{#1}}
\newcommand{\FunctionTok}[1]{\textcolor[rgb]{0.13,0.29,0.53}{\textbf{#1}}}
\newcommand{\ImportTok}[1]{#1}
\newcommand{\InformationTok}[1]{\textcolor[rgb]{0.56,0.35,0.01}{\textbf{\textit{#1}}}}
\newcommand{\KeywordTok}[1]{\textcolor[rgb]{0.13,0.29,0.53}{\textbf{#1}}}
\newcommand{\NormalTok}[1]{#1}
\newcommand{\OperatorTok}[1]{\textcolor[rgb]{0.81,0.36,0.00}{\textbf{#1}}}
\newcommand{\OtherTok}[1]{\textcolor[rgb]{0.56,0.35,0.01}{#1}}
\newcommand{\PreprocessorTok}[1]{\textcolor[rgb]{0.56,0.35,0.01}{\textit{#1}}}
\newcommand{\RegionMarkerTok}[1]{#1}
\newcommand{\SpecialCharTok}[1]{\textcolor[rgb]{0.81,0.36,0.00}{\textbf{#1}}}
\newcommand{\SpecialStringTok}[1]{\textcolor[rgb]{0.31,0.60,0.02}{#1}}
\newcommand{\StringTok}[1]{\textcolor[rgb]{0.31,0.60,0.02}{#1}}
\newcommand{\VariableTok}[1]{\textcolor[rgb]{0.00,0.00,0.00}{#1}}
\newcommand{\VerbatimStringTok}[1]{\textcolor[rgb]{0.31,0.60,0.02}{#1}}
\newcommand{\WarningTok}[1]{\textcolor[rgb]{0.56,0.35,0.01}{\textbf{\textit{#1}}}}
\usepackage{graphicx}
\makeatletter
\newsavebox\pandoc@box
\newcommand*\pandocbounded[1]{% scales image to fit in text height/width
  \sbox\pandoc@box{#1}%
  \Gscale@div\@tempa{\textheight}{\dimexpr\ht\pandoc@box+\dp\pandoc@box\relax}%
  \Gscale@div\@tempb{\linewidth}{\wd\pandoc@box}%
  \ifdim\@tempb\p@<\@tempa\p@\let\@tempa\@tempb\fi% select the smaller of both
  \ifdim\@tempa\p@<\p@\scalebox{\@tempa}{\usebox\pandoc@box}%
  \else\usebox{\pandoc@box}%
  \fi%
}
% Set default figure placement to htbp
\def\fps@figure{htbp}
\makeatother
\setlength{\emergencystretch}{3em} % prevent overfull lines
\providecommand{\tightlist}{%
  \setlength{\itemsep}{0pt}\setlength{\parskip}{0pt}}
\usepackage{bookmark}
\IfFileExists{xurl.sty}{\usepackage{xurl}}{} % add URL line breaks if available
\urlstyle{same}
\hypersetup{
  pdftitle={Comparing the Effectiveness of VR-Based and CBT Interventions for Autism Spectrum Disorder},
  pdfauthor={Brendan Flaherty (D00285274) \& Jeremiah Spillane (D00295490)},
  hidelinks,
  pdfcreator={LaTeX via pandoc}}

\title{Comparing the Effectiveness of VR-Based and CBT Interventions for
Autism Spectrum Disorder}
\author{Brendan Flaherty (D00285274) \& Jeremiah Spillane (D00295490)}
\date{2026-01-03}

\begin{document}
\maketitle

{
\setcounter{tocdepth}{2}
\tableofcontents
}
\section{Data Analytics for Immersive Environments
(COMPC9047)}\label{data-analytics-for-immersive-environments-compc9047}

\subsection{Investigating the Impact of Therapeutic Interventions on
Austistic Traits using the Autism Spectrum Quotient
(AQ)}\label{investigating-the-impact-of-therapeutic-interventions-on-austistic-traits-using-the-autism-spectrum-quotient-aq}

\textbf{Participants:} Brendan Flaherty (D00285274) \& Jeremiah Spillane
(D00295490) \textbf{Module:} Data Analytics for Immersive Environments
(COMPC9047) \textbf{Date:} 2026-01-03

\begin{center}\rule{0.5\linewidth}{0.5pt}\end{center}

\section{Abstract}\label{abstract}

\subsection{Aim and rationale}\label{aim-and-rationale}

This research paper examines changes in Autism Spectrum Quotient (AQ)
scores across two parameters. Firstly across time, that is pre-treatment
and post-treatment over a 12 week period, and secondly, across three
groups employing various approaches. The different approaches employed
are: 1) cognitive behavioral therapy (CBT), 2) virtual reality (VR), and
3) a control group. This study aims to answer whether CBT or VR-based
therapy lead to any statistically significant change in AQ scores
compared to no treatment, and whether or not one intervention is more
effective than the other.

\subsection{Participants and setting}\label{participants-and-setting}

The study involved a total of 300 adult participants (150 male, 150
female) aged between 12 and 40 years. Participants were randomly sampled
from a clinical population and randomly assigned to one of three equal
sized groups: a Cognitive Behavioral Therapy (CBT) group, a Virtual
Reality (VR) therapy group, and a control group receiving no
intervention.

\subsection{Experiment design}\label{experiment-design}

This study employed a randomised controlled, pre-test/post AQ tests,
mixed experimental design to examine the effectiveness of therapeutic
interventions on autistic traits over time (t = 12 weeks).

\subsection{Results gathering}\label{results-gathering}

Results were gathered by comparing pre- and post-treatment Autism
Spectrum Quotient (AQ) scores across participants using these
statistical tests, including paired t-tests, between group t-tests, and
z-tests, alongside descriptive statistics and 95\% confidence intervals
to quantify change over the 12-week period. This approach assessed
whether observed changes in AQ scores were statistically significant.

\subsection{Findings/implications}\label{findingsimplications}

The findings indicated a statistically notable reduction in AQ scores,
suggesting that therapeutic intervention was associated with a
measurable change in reported autistic traits across the sample.
However, the implications are limited by reliance on self report data
and the need to interpret statistical significance cautiously when
considering real world or clinical impact.

\section{Introduction}\label{introduction}

\subsection{Topic and Context}\label{topic-and-context}

As defined by the World Health Organisation (WHO) ``Autism spectrum
disorders (ASD) are a diverse group of conditions. They are
characterized by some degree of difficulty with social interaction and
communication. Other characteristics are atypical patterns of activities
and behaviors, such as difficulty with transition from one activity to
another, a focus on details and unusual reactions to sensations''.
Additionally, it has been suggested that ``Cognitive Behavior Therapy
(CBT) is beneficial to people with Autism Spectrum Disorder (ASD) but
that the method needs to be modified in relation to their cognitive
profile'' (Ekman E., et al., 2015). Virtual reality (VR) too has been
proposed to support those with ASD in educational settings, with
emotional regulation and more (Herrero, J.F. et al., 2025).

\subsection{Rationale}\label{rationale}

This research paper examines changes in Autism Spectrum Quotient (AQ)
scores across two parameters. Firstly across time, that is pre-treatment
and post-treatmnet over a 12 week period, and secondly, across three
groups employing various approaches. These groups employ different
approaches, namely 1) cognitive behavioral therapy (CBT) ; 2) virtual
reality (VR) and 3) a control group. This study aims to answer whether
CBT or VR-based therapy lead to a statistically significant change in AQ
scores compared to no treatment, and whether or not one intervention is
more effective than the other.

\subsection{Hypotesis}\label{hypotesis}

Hypothesis 1: After controlling for baseline AQ, there will be no group
differences in post-treatment (null-hypothesis).

Hypothesis 2: Both CBT and VR therapy groups will show significant
reductions in AQ scores from pre-treatment to post-treatment, while the
control group will not.

Hypothesis 3: The VR therapy group will show a statistically
significantly greater reduction in AQ scores from pre-treatment to
post-treatment than the CBT group.

\section{Method}\label{method}

\subsection{Design}\label{design}

A mixed-design experimental approach was employed, with time
(pre-treatment vs post-treatment) as a within-subjects factor and
treatment group (CBT, VR, control) as a between-subjects factor. The
primary dependent variable was the Autism Spectrum Quotient (AQ) score.
AQ measurements were collected at baseline and after a 12-week study
period.

\subsection{Materials}\label{materials}

Autistic traits were assessed using the Autism Spectrum Quotient (AQ), a
50-item self-report questionnaire with scores ranging from 0 to 50,
where higher scores indicate greater expression of autistic traits. The
AQ is widely used as a dimensional screening instrument and allows for
comparison of changes in trait expression over time.

\subsection{Procedure}\label{procedure}

Participants assigned to the CBT group received weekly 50-minute therapy
sessions focused on emotional and social regulation. Participants in the
VR therapy group engaged in weekly 50-minute VR-based sessions designed
to simulate real-world social interactions in a controlled and immersive
environment. The control group did not receive any therapeutic
intervention during the study period. AQ scores were collected from all
participants at the beginning of the study (pre-treatment) and again
after 12 weeks (post-treatment).

\section{Results}\label{results}

\subsubsection{**Data Cleaning*}\label{data-cleaning}

\begin{Shaded}
\begin{Highlighting}[]
\CommentTok{\# Adding librarys to notebook to run the code correctly}
\ControlFlowTok{if}\NormalTok{(}\SpecialCharTok{!}\FunctionTok{require}\NormalTok{(}\StringTok{"tidyverse"}\NormalTok{))}
  \FunctionTok{install.packages}\NormalTok{(}\StringTok{"tidyverse"}\NormalTok{)}
\end{Highlighting}
\end{Shaded}

\begin{verbatim}
## Loading required package: tidyverse
\end{verbatim}

\begin{verbatim}
## -- Attaching core tidyverse packages ------------------------ tidyverse 2.0.0 --
## v dplyr     1.1.4     v readr     2.1.6
## v forcats   1.0.1     v stringr   1.6.0
## v ggplot2   4.0.1     v tibble    3.3.0
## v lubridate 1.9.4     v tidyr     1.3.2
## v purrr     1.2.0     
## -- Conflicts ------------------------------------------ tidyverse_conflicts() --
## x dplyr::filter() masks stats::filter()
## x dplyr::lag()    masks stats::lag()
## i Use the conflicted package (<http://conflicted.r-lib.org/>) to force all conflicts to become errors
\end{verbatim}

\begin{Shaded}
\begin{Highlighting}[]
  \FunctionTok{library}\NormalTok{(tidyverse)}

\ControlFlowTok{if}\NormalTok{(}\SpecialCharTok{!}\FunctionTok{require}\NormalTok{(}\StringTok{"ggpubr"}\NormalTok{))}
  \FunctionTok{install.packages}\NormalTok{(}\StringTok{"ggpubr"}\NormalTok{)}
\end{Highlighting}
\end{Shaded}

\begin{verbatim}
## Loading required package: ggpubr
\end{verbatim}

\begin{Shaded}
\begin{Highlighting}[]
  \FunctionTok{library}\NormalTok{(ggpubr)}

\ControlFlowTok{if}\NormalTok{(}\SpecialCharTok{!}\FunctionTok{require}\NormalTok{(}\StringTok{"rstatix"}\NormalTok{))}
  \FunctionTok{install.packages}\NormalTok{(}\StringTok{"rstatix"}\NormalTok{)}
\end{Highlighting}
\end{Shaded}

\begin{verbatim}
## Loading required package: rstatix
## 
## Attaching package: 'rstatix'
## 
## The following object is masked from 'package:stats':
## 
##     filter
\end{verbatim}

\begin{Shaded}
\begin{Highlighting}[]
  \FunctionTok{library}\NormalTok{(rstatix)}

\ControlFlowTok{if}\NormalTok{(}\SpecialCharTok{!}\FunctionTok{require}\NormalTok{(}\StringTok{"knitr"}\NormalTok{))}
\FunctionTok{install.packages}\NormalTok{(}\StringTok{"knitr"}\NormalTok{)}
\end{Highlighting}
\end{Shaded}

\begin{verbatim}
## Loading required package: knitr
\end{verbatim}

\begin{Shaded}
\begin{Highlighting}[]
\FunctionTok{library}\NormalTok{(knitr)}


\CommentTok{\# Read the data frames of the .csv file}
\NormalTok{df }\OtherTok{\textless{}{-}} \FunctionTok{read\_csv}\NormalTok{(}\StringTok{"CA1\_Autism\_VR\_Study\_Data\_Final.csv"}\NormalTok{)}
\end{Highlighting}
\end{Shaded}

\begin{verbatim}
## Rows: 300 Columns: 6
## -- Column specification --------------------------------------------------------
## Delimiter: ","
## chr (2): Group, Gender
## dbl (4): ID, P1_Pre_Score, P2_Post_Score, Change_Score
## 
## i Use `spec()` to retrieve the full column specification for this data.
## i Specify the column types or set `show_col_types = FALSE` to quiet this message.
\end{verbatim}

\begin{Shaded}
\begin{Highlighting}[]
\CommentTok{\# Check the column names}
\FunctionTok{colnames}\NormalTok{(df)}
\end{Highlighting}
\end{Shaded}

\begin{verbatim}
## [1] "ID"            "Group"         "Gender"        "P1_Pre_Score" 
## [5] "P2_Post_Score" "Change_Score"
\end{verbatim}

\begin{Shaded}
\begin{Highlighting}[]
\CommentTok{\# Change column names to lowercase}
\NormalTok{df }\OtherTok{\textless{}{-}}\NormalTok{ df }\SpecialCharTok{\%\textgreater{}\%} \FunctionTok{rename\_with}\NormalTok{(tolower) }\SpecialCharTok{\%\textgreater{}\%} \FunctionTok{mutate}\NormalTok{(}\AttributeTok{group =} \FunctionTok{str\_trim}\NormalTok{(}\FunctionTok{tolower}\NormalTok{(group)))}
\CommentTok{\# Inspect Structure}
\FunctionTok{glimpse}\NormalTok{(df)}
\end{Highlighting}
\end{Shaded}

\begin{verbatim}
## Rows: 300
## Columns: 6
## $ id            <dbl> 1, 2, 3, 4, 5, 6, 7, 8, 9, 10, 11, 12, 13, 14, 15, 16, 1~
## $ group         <chr> "control", "control", "control", "control", "control", "~
## $ gender        <chr> "F", "F", "M", "F", "F", "F", "M", "F", "F", "F", "F", "~
## $ p1_pre_score  <dbl> 45, 47, 48, 45, 45, 41, 50, 47, 49, 45, 50, 46, 50, 47, ~
## $ p2_post_score <dbl> 37, 50, 46, 43, 47, 47, 50, 44, 50, 49, 46, 42, 50, 50, ~
## $ change_score  <dbl> 8, -3, 2, 2, -2, -6, 0, 3, -1, -4, 4, 4, 0, -3, 0, 6, 2,~
\end{verbatim}

\begin{Shaded}
\begin{Highlighting}[]
\FunctionTok{summary}\NormalTok{(}\FunctionTok{select}\NormalTok{(df, p1\_pre\_score, p2\_post\_score))}
\end{Highlighting}
\end{Shaded}

\begin{verbatim}
##   p1_pre_score   p2_post_score  
##  Min.   :32.00   Min.   :20.00  
##  1st Qu.:42.00   1st Qu.:28.00  
##  Median :45.00   Median :35.00  
##  Mean   :44.69   Mean   :34.92  
##  3rd Qu.:48.00   3rd Qu.:42.00  
##  Max.   :50.00   Max.   :50.00
\end{verbatim}

\begin{Shaded}
\begin{Highlighting}[]
\CommentTok{\# Check for missing values}
\FunctionTok{sapply}\NormalTok{(df, }\ControlFlowTok{function}\NormalTok{(x) }\FunctionTok{sum}\NormalTok{(}\FunctionTok{is.na}\NormalTok{(x)))}
\end{Highlighting}
\end{Shaded}

\begin{verbatim}
##            id         group        gender  p1_pre_score p2_post_score 
##             0             0             0             0             0 
##  change_score 
##             0
\end{verbatim}

\begin{Shaded}
\begin{Highlighting}[]
\CommentTok{\# Check for values outside the scope of the questionnaire {-} 0 {-} 50.1}
\FunctionTok{filter}\NormalTok{(df, p1\_pre\_score }\SpecialCharTok{\textless{}} \DecValTok{0} \SpecialCharTok{|}\NormalTok{ p1\_pre\_score }\SpecialCharTok{\textgreater{}} \FloatTok{50.1} \SpecialCharTok{|}\NormalTok{ p2\_post\_score }\SpecialCharTok{\textless{}} \DecValTok{0} \SpecialCharTok{|}\NormalTok{ p2\_post\_score }\SpecialCharTok{\textgreater{}} \FloatTok{50.1}\NormalTok{)}
\end{Highlighting}
\end{Shaded}

\begin{verbatim}
## # A tibble: 0 x 6
## # i 6 variables: id <dbl>, group <chr>, gender <chr>, p1_pre_score <dbl>,
## #   p2_post_score <dbl>, change_score <dbl>
\end{verbatim}

\begin{Shaded}
\begin{Highlighting}[]
\CommentTok{\# Drop incomplete rows}
\NormalTok{df }\OtherTok{\textless{}{-}}\NormalTok{ df }\SpecialCharTok{\%\textgreater{}\%} \FunctionTok{drop\_na}\NormalTok{(p1\_pre\_score, p2\_post\_score)}
\CommentTok{\# Identify change\_score (pre{-}post AQ)}
\NormalTok{df }\OtherTok{\textless{}{-}}\NormalTok{ df }\SpecialCharTok{\%\textgreater{}\%} \FunctionTok{mutate}\NormalTok{(}\AttributeTok{change\_score =}\NormalTok{ p1\_pre\_score }\SpecialCharTok{{-}}\NormalTok{ p2\_post\_score)}
\CommentTok{\# Confirm groups}
\FunctionTok{unique}\NormalTok{(df}\SpecialCharTok{$}\NormalTok{group)}
\end{Highlighting}
\end{Shaded}

\begin{verbatim}
## [1] "control" "cbt"     "vr"
\end{verbatim}

\begin{Shaded}
\begin{Highlighting}[]
\CommentTok{\# Check for summary of all columns}
\FunctionTok{summary}\NormalTok{(}\FunctionTok{select}\NormalTok{(df, p1\_pre\_score, p2\_post\_score, change\_score))}
\end{Highlighting}
\end{Shaded}

\begin{verbatim}
##   p1_pre_score   p2_post_score    change_score  
##  Min.   :32.00   Min.   :20.00   Min.   :-7.00  
##  1st Qu.:42.00   1st Qu.:28.00   1st Qu.: 3.00  
##  Median :45.00   Median :35.00   Median : 9.00  
##  Mean   :44.69   Mean   :34.92   Mean   : 9.77  
##  3rd Qu.:48.00   3rd Qu.:42.00   3rd Qu.:17.00  
##  Max.   :50.00   Max.   :50.00   Max.   :30.00
\end{verbatim}

\subsubsection{\texorpdfstring{\textbf{Descriptive
Statistics}}{Descriptive Statistics}}\label{descriptive-statistics}

\paragraph{\texorpdfstring{\textbf{Plots}}{Plots}}\label{plots}

\begin{Shaded}
\begin{Highlighting}[]
\CommentTok{\# BoxPlot created to show pre{-}treatment AQ scores}
\NormalTok{pre\_box }\OtherTok{\textless{}{-}} \FunctionTok{ggplot}\NormalTok{(df, }\FunctionTok{aes}\NormalTok{(}\AttributeTok{x =}\NormalTok{ group, }\AttributeTok{y =}\NormalTok{ p1\_pre\_score, }\AttributeTok{fill =}\NormalTok{ group)) }\SpecialCharTok{+} \FunctionTok{geom\_boxplot}\NormalTok{() }\SpecialCharTok{+} \FunctionTok{scale\_fill\_manual}\NormalTok{(}\AttributeTok{values =} \FunctionTok{c}\NormalTok{(}\StringTok{"skyblue"}\NormalTok{, }\StringTok{"lightgreen"}\NormalTok{, }\StringTok{"orange"}\NormalTok{)) }\SpecialCharTok{+} \FunctionTok{labs}\NormalTok{(}\AttributeTok{title =} \StringTok{"Pre{-}Treatment AQ Scores"}\NormalTok{,  }\AttributeTok{x =} \StringTok{"Group"}\NormalTok{, }\AttributeTok{y =} \StringTok{"AQ Score (0–50)"}\NormalTok{) }\SpecialCharTok{+} \FunctionTok{theme\_minimal}\NormalTok{(}\AttributeTok{base\_size =} \DecValTok{13}\NormalTok{) }\SpecialCharTok{+} \FunctionTok{theme}\NormalTok{(}\AttributeTok{plot.title =} \FunctionTok{element\_text}\NormalTok{(}\AttributeTok{hjust =} \FloatTok{0.5}\NormalTok{, }\AttributeTok{face =} \StringTok{"bold"}\NormalTok{))}

\CommentTok{\# BoxPlot created to show post{-}treatment AQ scores}
\NormalTok{post\_box }\OtherTok{\textless{}{-}} \FunctionTok{ggplot}\NormalTok{(df, }\FunctionTok{aes}\NormalTok{(}\AttributeTok{x =}\NormalTok{ group, }\AttributeTok{y =}\NormalTok{ p2\_post\_score, }\AttributeTok{fill =}\NormalTok{ group)) }\SpecialCharTok{+} \FunctionTok{geom\_boxplot}\NormalTok{() }\SpecialCharTok{+} \FunctionTok{scale\_fill\_manual}\NormalTok{(}\AttributeTok{values =} \FunctionTok{c}\NormalTok{(}\StringTok{"skyblue"}\NormalTok{, }\StringTok{"lightgreen"}\NormalTok{, }\StringTok{"orange"}\NormalTok{)) }\SpecialCharTok{+} \FunctionTok{labs}\NormalTok{(}\AttributeTok{title =} \StringTok{"Post{-}Treatment AQ Scores"}\NormalTok{,  }\AttributeTok{x =} \StringTok{"Group"}\NormalTok{, }\AttributeTok{y =} \StringTok{"AQ Score (0–50)"}\NormalTok{) }\SpecialCharTok{+} \FunctionTok{theme\_minimal}\NormalTok{(}\AttributeTok{base\_size =} \DecValTok{13}\NormalTok{) }\SpecialCharTok{+} \FunctionTok{theme}\NormalTok{(}\AttributeTok{plot.title =} \FunctionTok{element\_text}\NormalTok{(}\AttributeTok{hjust =} \FloatTok{0.5}\NormalTok{, }\AttributeTok{face =} \StringTok{"bold"}\NormalTok{))}

\CommentTok{\# Use ggpbr library to arrange histograms side{-}by{-}side to show difference pre and post treatment}
\FunctionTok{library}\NormalTok{(ggpubr)}
\FunctionTok{ggarrange}\NormalTok{(pre\_box, post\_box, }\AttributeTok{ncol =} \DecValTok{2}\NormalTok{, }\AttributeTok{common.legend =} \ConstantTok{TRUE}\NormalTok{, }\AttributeTok{legend =} \StringTok{"bottom"}\NormalTok{)}
\end{Highlighting}
\end{Shaded}

\pandocbounded{\includegraphics[keepaspectratio]{CA2_DA_Notebook_files/figure-latex/box plot-1.pdf}}

\begin{Shaded}
\begin{Highlighting}[]
\CommentTok{\# Pre{-}treatment AQ scores shows similar median. Post{-}Treatment AQ scores show VR Therapy group noticeably lower indicating greater improvement. Boxes show IQR and whiskers show outliers}
\end{Highlighting}
\end{Shaded}

\begin{Shaded}
\begin{Highlighting}[]
\CommentTok{\# Summarise and group mean score data by group for creating bar chart}
\NormalTok{avg\_dep\_scores }\OtherTok{\textless{}{-}}\NormalTok{ df }\SpecialCharTok{\%\textgreater{}\%} \FunctionTok{summarise}\NormalTok{(}\AttributeTok{mean\_pre =} \FunctionTok{mean}\NormalTok{(p1\_pre\_score), }\AttributeTok{mean\_post =} \FunctionTok{mean}\NormalTok{(p2\_post\_score), }\AttributeTok{.by =}\NormalTok{ group) }\SpecialCharTok{\%\textgreater{}\%} \FunctionTok{pivot\_longer}\NormalTok{(}\AttributeTok{cols =} \FunctionTok{c}\NormalTok{(mean\_pre, mean\_post), }\AttributeTok{names\_to =} \StringTok{"time"}\NormalTok{, }\AttributeTok{values\_to =} \StringTok{"mean\_score"}\NormalTok{) }\SpecialCharTok{\%\textgreater{}\%} \FunctionTok{mutate}\NormalTok{(}\AttributeTok{time =} \FunctionTok{recode}\NormalTok{(time, }\StringTok{"mean\_pre"} \OtherTok{=} \StringTok{"Pre{-}Treatment"}\NormalTok{, }\StringTok{"mean\_post"} \OtherTok{=} \StringTok{"Post{-}Treatment"}\NormalTok{))}

\CommentTok{\# Bar chart created to show mean AQ scores before and after treatment for each group}
\FunctionTok{ggplot}\NormalTok{(avg\_dep\_scores, }\FunctionTok{aes}\NormalTok{(}\AttributeTok{x =}\NormalTok{ group, }\AttributeTok{y =}\NormalTok{ mean\_score, }\AttributeTok{fill =}\NormalTok{ time)) }\SpecialCharTok{+} \FunctionTok{geom\_bar}\NormalTok{(}\AttributeTok{stat =} \StringTok{"identity"}\NormalTok{, }\AttributeTok{position =} \FunctionTok{position\_dodge}\NormalTok{()) }\SpecialCharTok{+} \FunctionTok{scale\_fill\_manual}\NormalTok{(}\AttributeTok{values =} \FunctionTok{c}\NormalTok{(}\StringTok{"orange"}\NormalTok{, }\StringTok{"skyblue"}\NormalTok{)) }\SpecialCharTok{+} \FunctionTok{labs}\NormalTok{(}\AttributeTok{title =} \StringTok{"Average AQ (Depression) Scores by Group"}\NormalTok{, }\AttributeTok{x =} \StringTok{"Therapy Group"}\NormalTok{, }\AttributeTok{y =} \StringTok{"Mean AQ Score (0{-}50)"}\NormalTok{ ) }\SpecialCharTok{+} \FunctionTok{theme\_minimal}\NormalTok{(}\AttributeTok{base\_size =} \DecValTok{13}\NormalTok{)}
\end{Highlighting}
\end{Shaded}

\pandocbounded{\includegraphics[keepaspectratio]{CA2_DA_Notebook_files/figure-latex/bar chart-1.pdf}}

\begin{Shaded}
\begin{Highlighting}[]
\CommentTok{\# Both CBT and VR Therapy groups show reduced mean AQ scores after treatment}
\end{Highlighting}
\end{Shaded}

\begin{Shaded}
\begin{Highlighting}[]
\CommentTok{\# Summarise and group mean score data by time for creating line graph}
\NormalTok{avg\_dep\_time }\OtherTok{\textless{}{-}}\NormalTok{ df }\SpecialCharTok{\%\textgreater{}\%} \FunctionTok{summarise}\NormalTok{(}\AttributeTok{mean\_pre =} \FunctionTok{mean}\NormalTok{(p1\_pre\_score), }\AttributeTok{mean\_post =} \FunctionTok{mean}\NormalTok{(p2\_post\_score), }\AttributeTok{.by =}\NormalTok{ group) }\SpecialCharTok{\%\textgreater{}\%} \FunctionTok{pivot\_longer}\NormalTok{(}\AttributeTok{cols =} \FunctionTok{c}\NormalTok{(mean\_pre, mean\_post), }\AttributeTok{names\_to =} \StringTok{"time"}\NormalTok{, }\AttributeTok{values\_to =} \StringTok{"mean\_score"}\NormalTok{) }\SpecialCharTok{\%\textgreater{}\%} \FunctionTok{mutate}\NormalTok{(}\AttributeTok{time =} \FunctionTok{recode}\NormalTok{(time, }\StringTok{"mean\_pre"} \OtherTok{=} \StringTok{"Pre{-}Treatment"}\NormalTok{, }\StringTok{"mean\_post"} \OtherTok{=} \StringTok{"Post{-}Treatment"}\NormalTok{))}

\CommentTok{\# Line graph showing mean AQ scores for each group pre and post treatment}
\FunctionTok{ggplot}\NormalTok{(avg\_dep\_time, }\FunctionTok{aes}\NormalTok{(}\AttributeTok{x =}\NormalTok{ time, }\AttributeTok{y =}\NormalTok{ mean\_score, }\AttributeTok{group =}\NormalTok{ group, }\AttributeTok{color =}\NormalTok{ group)) }\SpecialCharTok{+} \FunctionTok{geom\_line}\NormalTok{(}\AttributeTok{linewidth =} \FloatTok{1.2}\NormalTok{) }\SpecialCharTok{+} \FunctionTok{geom\_point}\NormalTok{(}\AttributeTok{size =} \DecValTok{13}\NormalTok{) }\SpecialCharTok{+} \FunctionTok{scale\_colour\_manual}\NormalTok{(}\AttributeTok{values =} \FunctionTok{c}\NormalTok{(}\StringTok{"orange"}\NormalTok{, }\StringTok{"skyblue"}\NormalTok{, }\StringTok{"lightgreen"}\NormalTok{)) }\SpecialCharTok{+} \FunctionTok{labs}\NormalTok{(}\AttributeTok{title =} \StringTok{"Change in Mean AQ Scores Over Time by Group"}\NormalTok{, }\AttributeTok{x =} \StringTok{"Time Point"}\NormalTok{, }\AttributeTok{y =} \StringTok{"Mean AQ Score (0{-}50)"}\NormalTok{ ) }\SpecialCharTok{+} \FunctionTok{theme\_minimal}\NormalTok{(}\AttributeTok{base\_size =} \DecValTok{13}\NormalTok{)}
\end{Highlighting}
\end{Shaded}

\pandocbounded{\includegraphics[keepaspectratio]{CA2_DA_Notebook_files/figure-latex/line-graph-1.pdf}}

\begin{Shaded}
\begin{Highlighting}[]
\CommentTok{\# Both CBT and VR Therapy groups downward trends from pre{-}treatment to post{-}treatment {-} VR Therapy shows greater reduction }
\end{Highlighting}
\end{Shaded}

\begin{Shaded}
\begin{Highlighting}[]
\CommentTok{\# Calculate pre{-}treatment AQ score distributions across the 3 groups}
\NormalTok{pre\_hist }\OtherTok{\textless{}{-}} \FunctionTok{ggplot}\NormalTok{(df, }\FunctionTok{aes}\NormalTok{(}\AttributeTok{x =}\NormalTok{ p1\_pre\_score, }\AttributeTok{fill =}\NormalTok{ group)) }\SpecialCharTok{+} \FunctionTok{geom\_histogram}\NormalTok{(}\AttributeTok{binwidth =} \DecValTok{2}\NormalTok{, }\AttributeTok{alpha =} \FloatTok{0.6}\NormalTok{, }\AttributeTok{position =} \StringTok{"identity"}\NormalTok{, }\AttributeTok{color =} \StringTok{"black"}\NormalTok{) }\SpecialCharTok{+} \FunctionTok{geom\_vline}\NormalTok{(}\AttributeTok{data =}\NormalTok{ df }\SpecialCharTok{\%\textgreater{}\%} \FunctionTok{group\_by}\NormalTok{(group) }\SpecialCharTok{\%\textgreater{}\%} \FunctionTok{summarise}\NormalTok{(}\AttributeTok{mean\_pre =} \FunctionTok{mean}\NormalTok{(p1\_pre\_score)), }\FunctionTok{aes}\NormalTok{(}\AttributeTok{xintercept =}\NormalTok{ mean\_pre, }\AttributeTok{colour =}\NormalTok{ group), }\AttributeTok{linetype =} \StringTok{"dashed"}\NormalTok{, }\AttributeTok{linewidth =} \DecValTok{1}\NormalTok{) }\SpecialCharTok{+}\FunctionTok{scale\_fill\_manual}\NormalTok{(}\AttributeTok{values =} \FunctionTok{c}\NormalTok{(}\StringTok{"orange"}\NormalTok{, }\StringTok{"skyblue"}\NormalTok{, }\StringTok{"lightgreen"}\NormalTok{)) }\SpecialCharTok{+} \FunctionTok{scale\_colour\_manual}\NormalTok{(}\AttributeTok{values =} \FunctionTok{c}\NormalTok{(}\StringTok{"orange"}\NormalTok{, }\StringTok{"skyblue"}\NormalTok{, }\StringTok{"lightgreen"}\NormalTok{)) }\SpecialCharTok{+} \FunctionTok{labs}\NormalTok{(}\AttributeTok{title =} \StringTok{"Pre{-}Treatment AQ Score Distribution"}\NormalTok{, }\AttributeTok{x =} \StringTok{"AQ Score (0 {-}50)"}\NormalTok{, }\AttributeTok{y =} \StringTok{"Frequency"}\NormalTok{) }\SpecialCharTok{+} \FunctionTok{theme\_minimal}\NormalTok{(}\AttributeTok{base\_size =} \DecValTok{13}\NormalTok{) }\SpecialCharTok{+} \FunctionTok{theme}\NormalTok{(}\AttributeTok{plot.title =} \FunctionTok{element\_text}\NormalTok{(}\AttributeTok{hjust =} \FloatTok{0.5}\NormalTok{, }\AttributeTok{face =} \StringTok{"bold"}\NormalTok{))}

\CommentTok{\# Calculate post{-}treatment AQ score distributions across the 3 groups}
\NormalTok{post\_hist }\OtherTok{\textless{}{-}} \FunctionTok{ggplot}\NormalTok{(df, }\FunctionTok{aes}\NormalTok{(}\AttributeTok{x =}\NormalTok{ p2\_post\_score, }\AttributeTok{fill =}\NormalTok{ group)) }\SpecialCharTok{+} \FunctionTok{geom\_histogram}\NormalTok{(}\AttributeTok{binwidth =} \DecValTok{2}\NormalTok{, }\AttributeTok{alpha =} \FloatTok{0.7}\NormalTok{, }\AttributeTok{colour =} \StringTok{"black"}\NormalTok{) }\SpecialCharTok{+} \FunctionTok{geom\_vline}\NormalTok{(}\AttributeTok{data =}\NormalTok{ df }\SpecialCharTok{\%\textgreater{}\%} \FunctionTok{group\_by}\NormalTok{(group) }\SpecialCharTok{\%\textgreater{}\%} \FunctionTok{summarise}\NormalTok{(}\AttributeTok{mean\_post =} \FunctionTok{mean}\NormalTok{(p2\_post\_score)), }\FunctionTok{aes}\NormalTok{(}\AttributeTok{xintercept =}\NormalTok{ mean\_post, }\AttributeTok{color =}\NormalTok{ group), }\AttributeTok{linetype =} \StringTok{"dashed"}\NormalTok{, }\AttributeTok{linewidth =} \DecValTok{1}\NormalTok{) }\SpecialCharTok{+} \FunctionTok{scale\_fill\_manual}\NormalTok{(}\AttributeTok{values =} \FunctionTok{c}\NormalTok{(}\StringTok{"orange"}\NormalTok{, }\StringTok{"skyblue"}\NormalTok{, }\StringTok{"lightgreen"}\NormalTok{)) }\SpecialCharTok{+} \FunctionTok{scale\_colour\_manual}\NormalTok{(}\AttributeTok{values =} \FunctionTok{c}\NormalTok{(}\StringTok{"orange"}\NormalTok{, }\StringTok{"skyblue"}\NormalTok{, }\StringTok{"lightgreen"}\NormalTok{)) }\SpecialCharTok{+} \FunctionTok{labs}\NormalTok{(}\AttributeTok{title =} \StringTok{"Post{-}Treatment AQ Score Distribution"}\NormalTok{, }\AttributeTok{x =} \StringTok{"AQ Score (0{-}50)"}\NormalTok{, }\AttributeTok{y =} \StringTok{"Frequency"}\NormalTok{) }\SpecialCharTok{+} \FunctionTok{theme\_minimal}\NormalTok{(}\AttributeTok{base\_size =} \DecValTok{13}\NormalTok{) }\SpecialCharTok{+} \FunctionTok{theme}\NormalTok{(}\AttributeTok{plot.title =} \FunctionTok{element\_text}\NormalTok{(}\AttributeTok{hjust =} \FloatTok{0.5}\NormalTok{, }\AttributeTok{face =} \StringTok{"bold"}\NormalTok{))}

\CommentTok{\# Use ggpbr library to arrange histograms side{-}by{-}side to show difference pre and post treatment}
\FunctionTok{library}\NormalTok{(ggpubr)}
\FunctionTok{ggarrange}\NormalTok{(pre\_hist, post\_hist, }\AttributeTok{ncol =} \DecValTok{2}\NormalTok{, }\AttributeTok{common.legend =} \ConstantTok{TRUE}\NormalTok{, }\AttributeTok{legend =} \StringTok{"bottom"}\NormalTok{)}
\end{Highlighting}
\end{Shaded}

\pandocbounded{\includegraphics[keepaspectratio]{CA2_DA_Notebook_files/figure-latex/histograms-1.pdf}}

\begin{Shaded}
\begin{Highlighting}[]
\CommentTok{\# All groups show similar AQ score pre{-}treatment. Post{-}treatment, both CBT and VR Therapy groups show lower AQ score indicating improvement}
\end{Highlighting}
\end{Shaded}

\begin{Shaded}
\begin{Highlighting}[]
\CommentTok{\# Scatterplot showing relationship between pre{-}treatment AQ scores and change in AQ scores post{-}treatment}
\FunctionTok{ggplot}\NormalTok{(df, }\FunctionTok{aes}\NormalTok{(}\AttributeTok{x =}\NormalTok{ p1\_pre\_score, }\AttributeTok{y =}\NormalTok{ change\_score, }\AttributeTok{colour =}\NormalTok{ group)) }\SpecialCharTok{+} \FunctionTok{geom\_point}\NormalTok{(}\AttributeTok{size =} \DecValTok{3}\NormalTok{, }\AttributeTok{alpha =} \FloatTok{0.7}\NormalTok{) }\SpecialCharTok{+} \FunctionTok{geom\_smooth}\NormalTok{(}\AttributeTok{method =} \StringTok{"lm"}\NormalTok{, }\AttributeTok{se =} \ConstantTok{FALSE}\NormalTok{, }\AttributeTok{linetype =} \StringTok{"dashed"}\NormalTok{) }\SpecialCharTok{+} \FunctionTok{scale\_colour\_manual}\NormalTok{(}\AttributeTok{values =} \FunctionTok{c}\NormalTok{(}\StringTok{"orange"}\NormalTok{, }\StringTok{"skyblue"}\NormalTok{, }\StringTok{"lightgreen"}\NormalTok{)) }\SpecialCharTok{+} \FunctionTok{labs}\NormalTok{(}\AttributeTok{title =} \StringTok{"Pre{-}Treatment AQ vs Change in Score"}\NormalTok{, }\AttributeTok{x =} \StringTok{"Pre{-}Treatment AQ Score"}\NormalTok{, }\AttributeTok{y =} \StringTok{"Change in AQ (Pre {-} Post, Positive = Improvement)"}\NormalTok{ ) }\SpecialCharTok{+} \FunctionTok{theme\_minimal}\NormalTok{()}
\end{Highlighting}
\end{Shaded}

\begin{verbatim}
## `geom_smooth()` using formula = 'y ~ x'
\end{verbatim}

\pandocbounded{\includegraphics[keepaspectratio]{CA2_DA_Notebook_files/figure-latex/scatterplot-1.pdf}}

\begin{Shaded}
\begin{Highlighting}[]
\CommentTok{\# A positive change indicates improvement {-} Participants with higher pre{-}treatment AQ scores tended to have a larger improvement}
\end{Highlighting}
\end{Shaded}

\subsubsection{\texorpdfstring{\textbf{Magnitude and direction of
results}}{Magnitude and direction of results}}\label{magnitude-and-direction-of-results}

\subsubsection{\texorpdfstring{\textbf{Statistical
tests}}{Statistical tests}}\label{statistical-tests}

\subsubsection{\texorpdfstring{\textbf{Inferential
Statistics}}{Inferential Statistics}}\label{inferential-statistics}

\subsubsection{\texorpdfstring{\textbf{Hypothesis
1}}{Hypothesis 1}}\label{hypothesis-1}

\textbf{Paired t-test results}

Paired t-tests were conducted to examine within-group changes in AQ
scores across the 12-week intervention period. The CBT group
demonstrated a statistically significant reduction in AQ scores, with a
mean decrease of 10.09 points (t = 22.9, p \textless{} .001). The VR
therapy group showed an even larger and more consistent improvement,
with a mean reduction of 18.62 points (t = 37.6, p \textless{} .001). In
contrast, the control group exhibited no significant change over time
(mean difference = 0.6; t = 1.59, p = 0.115). These findings support the
hypothesis that both CBT and VR therapy reduce AQ scores, while no
improvement occurs without intervention.

\begin{Shaded}
\begin{Highlighting}[]
\CommentTok{\# Assign and filter by group}
\NormalTok{cbt }\OtherTok{\textless{}{-}}\NormalTok{ df }\SpecialCharTok{\%\textgreater{}\%} \FunctionTok{filter}\NormalTok{(group }\SpecialCharTok{==} \StringTok{"cbt"}\NormalTok{)}
\NormalTok{vr }\OtherTok{\textless{}{-}}\NormalTok{ df }\SpecialCharTok{\%\textgreater{}\%} \FunctionTok{filter}\NormalTok{(group }\SpecialCharTok{==} \StringTok{"vr"}\NormalTok{)}
\NormalTok{control }\OtherTok{\textless{}{-}}\NormalTok{ df }\SpecialCharTok{\%\textgreater{}\%} \FunctionTok{filter}\NormalTok{ (group }\SpecialCharTok{==} \StringTok{"control"}\NormalTok{)}

\CommentTok{\# t{-}test by group}
\NormalTok{t\_cbt }\OtherTok{\textless{}{-}} \FunctionTok{t.test}\NormalTok{(cbt}\SpecialCharTok{$}\NormalTok{p1\_pre\_score, cbt}\SpecialCharTok{$}\NormalTok{p2\_post\_score, }\AttributeTok{paired =} \ConstantTok{TRUE}\NormalTok{)}
\NormalTok{t\_vr }\OtherTok{\textless{}{-}} \FunctionTok{t.test}\NormalTok{(vr}\SpecialCharTok{$}\NormalTok{p1\_pre\_score, vr}\SpecialCharTok{$}\NormalTok{p2\_post\_score, }\AttributeTok{paired =} \ConstantTok{TRUE}\NormalTok{)}
\NormalTok{t\_control }\OtherTok{\textless{}{-}} \FunctionTok{t.test}\NormalTok{(control}\SpecialCharTok{$}\NormalTok{p1\_pre\_score, control}\SpecialCharTok{$}\NormalTok{p2\_post\_score, }\AttributeTok{paired =} \ConstantTok{TRUE}\NormalTok{)}

\CommentTok{\# Output results of t{-}tests in list}
\FunctionTok{list}\NormalTok{(}\AttributeTok{CBT =}\NormalTok{ t\_cbt, }\AttributeTok{VR =}\NormalTok{ t\_vr, }\AttributeTok{CONTROL =}\NormalTok{ t\_control)}
\end{Highlighting}
\end{Shaded}

\begin{verbatim}
## $CBT
## 
##  Paired t-test
## 
## data:  cbt$p1_pre_score and cbt$p2_post_score
## t = 22.899, df = 99, p-value < 2.2e-16
## alternative hypothesis: true mean difference is not equal to 0
## 95 percent confidence interval:
##   9.215681 10.964319
## sample estimates:
## mean difference 
##           10.09 
## 
## 
## $VR
## 
##  Paired t-test
## 
## data:  vr$p1_pre_score and vr$p2_post_score
## t = 37.633, df = 99, p-value < 2.2e-16
## alternative hypothesis: true mean difference is not equal to 0
## 95 percent confidence interval:
##  17.63826 19.60174
## sample estimates:
## mean difference 
##           18.62 
## 
## 
## $CONTROL
## 
##  Paired t-test
## 
## data:  control$p1_pre_score and control$p2_post_score
## t = 1.5887, df = 99, p-value = 0.1153
## alternative hypothesis: true mean difference is not equal to 0
## 95 percent confidence interval:
##  -0.1493572  1.3493572
## sample estimates:
## mean difference 
##             0.6
\end{verbatim}

\subsubsection{\texorpdfstring{\textbf{Hypothesis
2}}{Hypothesis 2}}\label{hypothesis-2}

\textbf{Between group t-test results}

Between-group comparisons further demonstrated that both treatment
conditions were significantly more effective than the control group.
Participants receiving VR therapy showed substantially greater
improvement than controls (t = −28.95, p \textless{} .001), as did those
receiving CBT (t = 16.35, p \textless{} .001). Direct comparison between
the two interventions revealed that VR therapy produced significantly
greater reductions in AQ scores than CBT (t = −12.38, p \textless{}
.001), indicating superior effectiveness of the VR-based intervention.

\begin{Shaded}
\begin{Highlighting}[]
\CommentTok{\# Assign and filter by groups {-} VR \& Control, CBT \& Control, VR and CBT}
\NormalTok{vr\_control }\OtherTok{\textless{}{-}}\NormalTok{ df }\SpecialCharTok{\%\textgreater{}\%} \FunctionTok{filter}\NormalTok{(group }\SpecialCharTok{\%in\%} \FunctionTok{c}\NormalTok{(}\StringTok{"vr"}\NormalTok{, }\StringTok{"control"}\NormalTok{))}
\NormalTok{cbt\_control }\OtherTok{\textless{}{-}}\NormalTok{ df }\SpecialCharTok{\%\textgreater{}\%} \FunctionTok{filter}\NormalTok{(group }\SpecialCharTok{\%in\%} \FunctionTok{c}\NormalTok{(}\StringTok{"cbt"}\NormalTok{, }\StringTok{"control"}\NormalTok{))}
\NormalTok{vr\_cbt }\OtherTok{\textless{}{-}}\NormalTok{ df }\SpecialCharTok{\%\textgreater{}\%} \FunctionTok{filter}\NormalTok{(group }\SpecialCharTok{\%in\%} \FunctionTok{c}\NormalTok{(}\StringTok{"vr"}\NormalTok{, }\StringTok{"cbt"}\NormalTok{))}

\CommentTok{\# t{-}test between groups {-} VR vs Control | CBT vs Control | VR vs CBT }
\NormalTok{t\_vr\_control }\OtherTok{\textless{}{-}} \FunctionTok{t.test}\NormalTok{(change\_score }\SpecialCharTok{\textasciitilde{}}\NormalTok{ group, }\AttributeTok{data =}\NormalTok{ vr\_control, }\AttributeTok{var.equal =} \ConstantTok{TRUE}\NormalTok{)}
\NormalTok{t\_cbt\_control }\OtherTok{\textless{}{-}} \FunctionTok{t.test}\NormalTok{(change\_score }\SpecialCharTok{\textasciitilde{}}\NormalTok{ group, }\AttributeTok{data =}\NormalTok{ cbt\_control, }\AttributeTok{var.equal =} \ConstantTok{TRUE}\NormalTok{)}
\NormalTok{t\_vr\_cbt }\OtherTok{\textless{}{-}} \FunctionTok{t.test}\NormalTok{(change\_score }\SpecialCharTok{\textasciitilde{}}\NormalTok{ group, }\AttributeTok{data =}\NormalTok{ vr\_cbt, }\AttributeTok{var.equal =} \ConstantTok{TRUE}\NormalTok{)}

\CommentTok{\# Output results of between group t{-}tests in list}
\FunctionTok{list}\NormalTok{(}\AttributeTok{VR\_vs\_CONTROL =}\NormalTok{ t\_vr\_control, }\AttributeTok{CBT\_vs\_CONTROL =}\NormalTok{ t\_cbt\_control, }\AttributeTok{VR\_vs\_CBT =}\NormalTok{ t\_vr\_cbt)}
\end{Highlighting}
\end{Shaded}

\begin{verbatim}
## $VR_vs_CONTROL
## 
##  Two Sample t-test
## 
## data:  change_score by group
## t = -28.951, df = 198, p-value < 2.2e-16
## alternative hypothesis: true difference in means between group control and group vr is not equal to 0
## 95 percent confidence interval:
##  -19.24746 -16.79254
## sample estimates:
## mean in group control      mean in group vr 
##                  0.60                 18.62 
## 
## 
## $CBT_vs_CONTROL
## 
##  Two Sample t-test
## 
## data:  change_score by group
## t = 16.353, df = 198, p-value < 2.2e-16
## alternative hypothesis: true difference in means between group cbt and group control is not equal to 0
## 95 percent confidence interval:
##   8.345572 10.634428
## sample estimates:
##     mean in group cbt mean in group control 
##                 10.09                  0.60 
## 
## 
## $VR_vs_CBT
## 
##  Two Sample t-test
## 
## data:  change_score by group
## t = -12.875, df = 198, p-value < 2.2e-16
## alternative hypothesis: true difference in means between group cbt and group vr is not equal to 0
## 95 percent confidence interval:
##  -9.836548 -7.223452
## sample estimates:
## mean in group cbt  mean in group vr 
##             10.09             18.62
\end{verbatim}

\subsubsection{\texorpdfstring{\textbf{Hypothesis
3}}{Hypothesis 3}}\label{hypothesis-3}

\textbf{z-test results}

A z-test assessing overall change across all participants indicated a
significant mean reduction in AQ scores of 9.77 points (z = 19.72, p
\textless{} .001). The 95\% confidence interval (8.8 to 10.74) was
narrow, suggesting a consistent treatment effect across the sample. This
result confirms that the observed improvements were highly unlikely to
be due to chance.

\begin{Shaded}
\begin{Highlighting}[]
\CommentTok{\# Calculate the sample mean of change score}
\NormalTok{mean\_change }\OtherTok{\textless{}{-}} \FunctionTok{mean}\NormalTok{(df}\SpecialCharTok{$}\NormalTok{change\_score)}
\CommentTok{\# Calculate the standard deviation of change score}
\NormalTok{sd\_change }\OtherTok{\textless{}{-}} \FunctionTok{sd}\NormalTok{(df}\SpecialCharTok{$}\NormalTok{change\_score)}
\CommentTok{\# Calculate the number of participants}
\NormalTok{n }\OtherTok{\textless{}{-}} \FunctionTok{nrow}\NormalTok{(df)}
\CommentTok{\# Calculate Z{-}value using the formuala}
\NormalTok{z\_value }\OtherTok{\textless{}{-}}\NormalTok{ (mean\_change }\SpecialCharTok{{-}} \DecValTok{0}\NormalTok{) }\SpecialCharTok{/}\NormalTok{ (sd\_change }\SpecialCharTok{/} \FunctionTok{sqrt}\NormalTok{(n))}
\CommentTok{\# Calculate p{-}value using the Z distribution}
\NormalTok{p\_value }\OtherTok{\textless{}{-}} \DecValTok{2} \SpecialCharTok{*}\NormalTok{ (}\DecValTok{1} \SpecialCharTok{{-}} \FunctionTok{pnorm}\NormalTok{(}\FunctionTok{abs}\NormalTok{(z\_value)))}
\CommentTok{\# Calculate the Confidence Interval around the mean}
\NormalTok{error\_margin }\OtherTok{\textless{}{-}} \FunctionTok{qnorm}\NormalTok{(}\FloatTok{0.975}\NormalTok{) }\SpecialCharTok{*}\NormalTok{ (sd\_change }\SpecialCharTok{/} \FunctionTok{sqrt}\NormalTok{(n))}
\CommentTok{\# Calculate the Lower limits of the CI}
\NormalTok{ci\_lower }\OtherTok{\textless{}{-}}\NormalTok{ mean\_change }\SpecialCharTok{{-}}\NormalTok{ error\_margin}
\CommentTok{\# Calculate the Upper limits of the CI}
\NormalTok{ci\_upper }\OtherTok{\textless{}{-}}\NormalTok{ mean\_change }\SpecialCharTok{+}\NormalTok{ error\_margin}
\CommentTok{\# Create a table for the results}
\FunctionTok{data.frame}\NormalTok{(}\AttributeTok{Mean\_Change =} \FunctionTok{round}\NormalTok{(mean\_change, }\DecValTok{2}\NormalTok{), }\AttributeTok{Z\_Value =} \FunctionTok{round}\NormalTok{(z\_value, }\DecValTok{2}\NormalTok{), }\AttributeTok{P\_Value =} \FunctionTok{round}\NormalTok{(p\_value, }\DecValTok{4}\NormalTok{), }\AttributeTok{CI\_Lower =} \FunctionTok{round}\NormalTok{(ci\_lower, }\DecValTok{2}\NormalTok{), }\AttributeTok{CI\_Upper =} \FunctionTok{round}\NormalTok{(ci\_upper, }\DecValTok{2}\NormalTok{))}
\end{Highlighting}
\end{Shaded}

\begin{verbatim}
##   Mean_Change Z_Value P_Value CI_Lower CI_Upper
## 1        9.77   19.72       0      8.8    10.74
\end{verbatim}

\section{Discussion}\label{discussion}

Overall, the findings provide strong evidence that both CBT and VR-based
therapy significantly reduce AQ scores, with VR therapy yielding the
largest and most consistent improvements, while the control group showed
no significant change. These results support the study hypotheses and
highlight the potential of immersive VR therapy as an effective
intervention for reducing autistic traits. The findings indicated a
statistically notable reduction in AQ scores, suggesting that
therapeutic intervention was associated with a measurable change in
reported autistic traits across the sample.

\paragraph{Limitations}\label{limitations}

These findings must be considered with limitations which include, and
are not limited to, (over-)reliance on self-report data and the need to
interpret statistical significance cautiously when considering
real-world or clinical cases.

\begin{center}\rule{0.5\linewidth}{0.5pt}\end{center}

\section{References}\label{references}

Baus, Oliver \& Bouchard, Stéphane. (2014). Moving from Virtual Reality
Exposure-Based Therapy to Augmented Reality Exposure-Based Therapy: A
Review. Frontiers in human neuroscience. 8. 112.
10.3389/fnhum.2014.00112. (Accessed: 3 October 2023)

Ekman E. \& Hiltunen A. J. (2015). Modified CBT using visualization for
Autism Spectrum Disorder (ASD), anxiety and avoidance behavior -- a
quasi-experimental open pilot study. Scandinavian Journal of Psychology,
56, 641--648.

Freeman, D. et al.~(2017) Virtual reality in the assessment,
understanding, and treatment of mental health disorders: Psychological
medicine, Cambridge Core (Accessed: 6 October 2023)

Maples-Keller, J. L., Bunnell, B. E., Kim, S. J., \& Rothbaum, B. O.
(2017). The Use of Virtual Reality Technology in the Treatment of
Anxiety and Other Psychiatry Disorders. Harvard Review of Psychiatry,
25(3), 103-113. (Accessed: 4 October 2023)

Maskey, M., Rodgers, J., Ingham, B., Freeston, M., Evans, G., Labus, M.
and Parr, J.R. (2019). Using Virtual Reality Environments to Augment
Cognitive Behavioral Therapy for Fears and Phobias in Autistic Adults.
Autism in Adulthood, 1(2), pp.134--145.
\url{doi:https://doi.org/10.1089/aut.2018.0019}.

De Luca, R., Leonardi, S., Portaro, S., Le Cause, M., De Domenico, C.,
Colucci, P.V., Pranio, F., Bramanti, P. and Calabrò, R.S. (2019).
Innovative use of virtual reality in autism spectrum disorder: A
case-study. Applied Neuropsychology: Child, 10(1), pp.90--100.
\url{doi:https://doi.org/10.1080/21622965.2019.1610964}.

Herrero, J.F., Lorenzo, G. An immersive virtual reality educational
intervention on people with autism spectrum disorders (ASD) for the
development of communication skills and problem solving. Educ Inf
Technol 25, 1689--1722 (2020).
\url{https://doi-org.mtu.idm.oclc.org/10.1007/s10639-019-10050-0}

World Health Organization. (2025). Autism. World Health Organization.
\url{https://www.who.int/news-room/fact-sheets/detail/autism-spectrum-disorders}

\end{document}
